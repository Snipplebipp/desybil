\documentclass{article}
\title{Feasibility Analysis - DeSybil}
\author{Dolan Murvihill}
\date{}

\begin{document}
\section{LTE}
\section{GSM}
\section{CDMA}
\section{Wi-Fi}
\section{Bluetooth}
While a Bluetooth interface has a 48-bit MAC address, it is not usually
transmitted as part of a reply to a discovery request; instead, Bluetooth
devices send a human-readable hostname. This could be useful in cases where the
owner of the device has not set a Bluetooth name, since the default Bluetooth
identifier would positively identify the model of the phone.

Beginning in 2003, the public began noticing unsolicited messages on their
Bluetooth-enabled devices. These messages appeared when a Bluetooth-enabled
device paired with another Bluetooth device nearby and sent an unwanted message
to its owner. \emph{Bluejacking}, as it was called, was often used for guerrilla
marketing, but it might also be used in a phishing attack - an attacker could
use bluejacking to send a link to the owner of a nearby Bluetooth device and
then look for traffic that resembles web traffic on Wi-Fi and other interfaces.
%citation needed on this paragraph.

The prospect of running a De-Sybil attack on a Bluetooth device suffers from a
huge drawback: almost all Bluetooth devices disable inquiry responses by
default. %citation needed
We should conduct a passive observation of Bluetooth transmission behavior in a
crowded area to get a sense for how many devices a Bluetooth attack may be
possible against.

\section{NFC}
